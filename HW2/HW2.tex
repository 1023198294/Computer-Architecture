\documentclass[paper=a4, fontsize=11pt]{scrartcl} % A4 paper and 11pt font size
\usepackage{../usfassignment}
\usepackage{../mips}
\settitle{Assignment 2}
\setauthor{Wanzhang Sheng}
\setcourse{CS315: Computer Architecture}

\begin{document}

\maketitle % Print the title

% -----------------------------------------------------------------------------
% PROBLEM 1
% -----------------------------------------------------------------------------
\section{1.12}
\begin{fancyquotes}
  Section 1.10 cites as a pitfall the utilization of a subset of the
  performance equation as a performance metric. To illustrate this,
  consider the following two processors. $P1$ has a clock rate of
  $\SI{4}{\giga\hertz}$, average CPI of $0.9$, and requires the
  execution of $\SI{5.0E9}{}$ instructions. $P2$ has a clock rate of
  $\SI{3}{\giga\hertz}$, an average CPI of $\SI{0.75}{}$, and requires
  the execution of $\SI{1.0E9}{}$ instructions.
\end{fancyquotes}

\subsection{1.12.3}
\begin{fancyquotes}
  A common fallacy is to use MIPS (millions of instructions per
  second) to compare the performance of two different processors, and
  consider that the processor with the largest MIPS has the largest
  performance. Check if this is true for $P1$ and $P2$.
\end{fancyquotes}

$$\text{MPIS} = \text{Instructions} / (\text{Exec time}\times\SI{e6}{})
= \text{Clock rate} /(\text{CPI}\times\SI{e6}{})$$
$$\text{Performance} = \text{Exec time}^{-1}
= \text{MIPS}\times\SI{e6}{} / \text{Instructions}$$

\begin{table}[hp]
  \centering
  \begin{tabular}[hp]{ccc}
    & MIPS & Performance\\
    \toprule
    P1 & $\SI{4}{\giga\hertz}/0.9=4444$  & $4.444/5.0 = 0.8889$ \\
    P2 & $\SI{3}{\giga\hertz}/0.75=4000$ & $4 / 1.0 = 4$
  \end{tabular}
  \caption{MIPS vs Performance}
\end{table}

$P1$ is slower but has a higher MIPS.

\subsection{1.12.4}
\begin{fancyquotes}
  Another common performance figure is MFLOPS (millions of
  floating-point operations per second), defined as
  $$\text{MFLOPS} = \text{No.FP operations} / (\text{execution time}\times\SI{1e6}{})$$
  but this figure has the same problems as MIPS. Assume that $40\%$ of
  the instructions executed on both $P1$ and $P2$ are floating-point
  instructions. Find the MFLOPS figures for the programs.
\end{fancyquotes}

$$\text{MFLOPS} = 40\%\times\text{Instructions} /
(\text{Exec time}\times\SI{e6}{})
= 40\%\times\text{MIPS}$$

\begin{enumerate}
\item{P1:} $40\%\times 4444 = 1778$
\item{P2:} $40\%\times 4000 = 1600$
\end{enumerate}

\pagebreak

% -----------------------------------------------------------------------------
% PROBLEM 2
% -----------------------------------------------------------------------------
\section{}
\begin{fancyquotes}
  Show how the pseudoinstruction \textit{bgt} can be implemented using
  ``core'' MIPS instructions --- instructions in the first column on
  page 1 of the MIPS Green Sheet. Recollect that the syntax of bgt is
  \textit{bgt \$reg0, \$reg1, gt\_lab}

  If the contents of \textit{\$reg0} are greater than the contents of
  \textit{\$reg1} control goes to the instruction with label
  \textit{gt\_lab}. If the contents of \textit{\$reg0} are less than
  or equal to \textit{\$reg1}, control goes to the instruction
  following the \textit{bgt} instruction.
\end{fancyquotes}

\lstinputlisting[language={[mips]Assembler}]{bgt.s}

\pagebreak

% -----------------------------------------------------------------------------
% PROBLEM 3
% -----------------------------------------------------------------------------
\section{}
\begin{fancyquotes}
  Write a MIPS assembly language program that reads in three ints and
  subtracts the last int from the sum of the first two ints. When it's
  done with the calculations it should print the result.
\end{fancyquotes}

\lstinputlisting[language={[mips]Assembler}]{sub.s}

\pagebreak

% -----------------------------------------------------------------------------
% PROBLEM 4
% -----------------------------------------------------------------------------
\section{}
\begin{fancyquotes}
  Write a MIPS assembly language program that reads in two ints and
  compares them. Depending on the outcome of the comparison it should
  print one of the messages ``First is greater than second,''
  ``They're equal,'' or ``First is less than second.''
\end{fancyquotes}

\lstinputlisting[language={[mips]Assembler}]{comp.s}

\pagebreak

% -----------------------------------------------------------------------------
% PROBLEM 5
% -----------------------------------------------------------------------------
\section{}
\begin{fancyquotes}
  This C program finds the nth Fibonacci number. Translate the program
  into MIPS assembly language.
\end{fancyquotes}

C program:

\lstinputlisting[language=C, firstline=22, lastline=37]{fibo.c}

MIPS program:

\lstinputlisting[language={[mips]Assembler}]{fibo.s}

\pagebreak

\end{document}
