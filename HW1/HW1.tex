\documentclass[paper=a4, fontsize=11pt]{scrartcl} % A4 paper and 11pt font size
\usepackage{./../usfassignment}
\usepackage{siunitx}
\settitle{Assignment 1}
\setauthor{Wanzhang Sheng}
\setcourse{CS315: Computer Architecture}

\begin{document}

\maketitle % Print the title

% -----------------------------------------------------------------------------
% PROBLEM 1
% -----------------------------------------------------------------------------
\section{}

\begin{fancyquotes}
  1.5 [4] <§1.6>

  Consider three different processors $P1$, $P2$, and $P3$ executing
  the same instruction set. $P1$ has a $\SI{3}{\giga\hertz}$ clock
  rate and a CPI of $1.5$. $P2$ has a $\SI{2.5}{\giga\hertz}$ clock
  rate and a CPI of $1.0$. $P3$ has a $\SI{4.0}{\giga\hertz}$ clock
  rate and has a CPI of $2.2$.
 \end{fancyquotes}

\begin{enumerate}
\item
  \begin{fancyquotes}
    Which processor has the highest performance expressed in
    instructions per second?
  \end{fancyquotes}

  $$\text{Instructions} / \text{second} = \frac{\text{clock rate}}{\text{CPI}}$$

  \begin{itemize}
  \item{P1:} $\SI{3.0}{\giga\hertz} / 1.5 = \SI{2e9}{}$
  \item{P2:} $\SI{2.5}{\giga\hertz} / 1.0 = \SI{2.5e9}{}$
  \item{P3:} $\SI{4.0}{\giga\hertz} / 2.2 = \SI{1.818e9}{}$
  \end{itemize}

  So $P2$ has the hightest performance in instructions per second.

\item
  \begin{fancyquotes}
    If the processors each execute a program in $\SI{10}{\second}$,
    find the number of cycles and the number of instructions.
  \end{fancyquotes}

  $$\text{cycles} = \text{clock rate} \times \text{time in seconds}$$

  \begin{itemize}
  \item{P1:} $\SI{3.0}{\giga\hertz}\times\SI{10}{\second} = \SI{3e10}{cycles}$
  \item{P2:} $\SI{2.5}{\giga\hertz}\times\SI{10}{\second} = \SI{2.5e10}{cycles}$
  \item{P3:} $\SI{4.0}{\giga\hertz}\times\SI{10}{\second} = \SI{4e10}{cycles}$
  \end{itemize}

  $$\text{instructions} = \frac{\text{cycles}}{\text{CPI}}$$

  \begin{itemize}
  \item{P1:} $\SI{3e10}{} / \SI{1.5}{} = \SI{2e10}{instructions}$
  \item{P2:} $\SI{2.5e10}{} / \SI{1.0}{} = \SI{2.5e10}{instructions}$
  \item{P3:} $\SI{4e10}{} / \SI{2.2}{} = \SI{1.818e10}{instructions}$
  \end{itemize}

\item
  \begin{fancyquotes}
    We are trying to reduce the time by 30\% but this leads to an
    increase of 20\% in the CPI\@. What clock rate should we have to get
    this time reduction?
  \end{fancyquotes}

  $$\text{instructions} = \frac{\text{cycles}}{\text{CPI}}
  = \frac{\text{clock rate}\times\text{time}}{\text{CPI}}$$
  $$\text{clock rate}
  = \frac{\text{instructions}\times\text{CPI}}{\text{time}}$$
  $$\text{clock rate}^{\prime}
  = \frac{\text{instructions}\times\text{CPI}^{\prime}}{\text{time}^{\prime}}$$
  $$\text{CPI}^{\prime} = 1.2\text{CPI}$$
  $$\text{time}^{\prime} = 0.7\text{time}$$

  So:
  $$\frac{\text{clock rate}^{\prime}}{\text{clock rate}}
  = \frac{\text{CPI}^{\prime}}{\text{CPI}}\times
  \frac{\text{time}^{\prime}}{\text{time}}
  = 1.2 / 0.7 = \frac{12}{7}$$

  For processors:
  \begin{itemize}
  \item{P1:} $\SI{3.0}{\giga\hertz}\times\frac{12}{7} =
    \SI{5.1429}{\giga\hertz}$
  \item{P2:} $\SI{2.5}{\giga\hertz}\times\frac{12}{7} =
    \SI{4.2857}{\giga\hertz}$
  \item{P3:} $\SI{4.0}{\giga\hertz}\times\frac{12}{7} =
    \SI{6.8571}{\giga\hertz}$
  \end{itemize}

\end{enumerate}


\pagebreak

% -----------------------------------------------------------------------------
% PROBLEM 2
% -----------------------------------------------------------------------------
\section{}

\begin{fancyquotes}
  1.6 [20] <§1.6> Consider two different implementations of the same
  instruction set architecture. The instructions can be divided into
  four classes according to their CPI (class A, B, C, and D). P1 with
  a clock rate of 2.5 GHz and CPIs of 1, 2, 3, and 3, and P2 with a
  clock rate of 3 GHz and CPIs of 2, 2, 2, and 2.

  Given a program with a dynamic instruction count of 1.0E6
  instructions divided into classes as follows: 10\% class A, 20\%
  class B, 50\% class C, and 20\% class D, which implementation is
  faster?

  a. What is the global CPI for each implementation?
  b. Find the clock cycles required in both cases.”
\end{fancyquotes}

\pagebreak

% -----------------------------------------------------------------------------
% PROBLEM 3
% -----------------------------------------------------------------------------
\section{}

\begin{fancyquotes}
  1.7 [15] <§1.6> Compilers can have a profound impact on the
  performance of an application. Assume that for a program, compiler A
  results in a dynamic instruction count of 1.0E9 and has an execution
  time of 1.1 s, while compiler B results in a dynamic instruction
  count of 1.2E9 and an execution time of 1.5 s.

  a. Find the average CPI for each program given that the processor
  has a clock cycle time of 1 ns.
  b. Assume the compiled programs run on two different processors. If
  the execution times on the two processors are the same, how much
  faster is the clock of the processor running compiler A’s code
  versus the clock of the processor running compiler B’s code?
  c. A new compiler is developed that uses only 6.0E8 instructions and
  has an average CPI of 1.1. What is the speedup of using this new
  compiler versus using compiler A or B on the original processor?”
\end{fancyquotes}

\end{document}
