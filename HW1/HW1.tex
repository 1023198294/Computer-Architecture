\documentclass[paper=a4, fontsize=11pt]{scrartcl} % A4 paper and 11pt font size
\usepackage{./../usfassignment}
\usepackage{siunitx}
\settitle{Assignment 1}
\setauthor{Wanzhang Sheng}
\setcourse{CS315: Computer Architecture}

\begin{document}

\maketitle % Print the title

% -----------------------------------------------------------------------------
% PROBLEM 1
% -----------------------------------------------------------------------------
\section{}

\begin{fancyquotes}
  1.5 [4] <§1.6>

  Consider three different processors $P1$, $P2$, and $P3$ executing
  the same instruction set. $P1$ has a $\SI{3}{\giga\hertz}$ clock
  rate and a CPI of $1.5$. $P2$ has a $\SI{2.5}{\giga\hertz}$ clock
  rate and a CPI of $1.0$. $P3$ has a $\SI{4.0}{\giga\hertz}$ clock
  rate and has a CPI of $2.2$.
\end{fancyquotes}

\begin{enumerate}
\item
  \begin{fancyquotes}
    Which processor has the highest performance expressed in
    instructions per second?
  \end{fancyquotes}

  $$\text{Instructions} / \text{second} = \frac{\text{clock rate}}{\text{CPI}}$$

  \begin{itemize}
  \item{P1:} $\SI{3.0}{\giga\hertz} / 1.5 = \SI{2e9}{}$
  \item{P2:} $\SI{2.5}{\giga\hertz} / 1.0 = \SI{2.5e9}{}$
  \item{P3:} $\SI{4.0}{\giga\hertz} / 2.2 = \SI{1.818e9}{}$
  \end{itemize}

  So $P2$ has the hightest performance in instructions per second.

\item
  \begin{fancyquotes}
    If the processors each execute a program in $\SI{10}{\second}$,
    find the number of cycles and the number of instructions.
  \end{fancyquotes}

  $$\text{cycles} = \text{clock rate} \times \text{time in seconds}$$

  \begin{itemize}
  \item{P1:} $\SI{3.0}{\giga\hertz}\times\SI{10}{\second} = \SI{3e10}{cycles}$
  \item{P2:} $\SI{2.5}{\giga\hertz}\times\SI{10}{\second} = \SI{2.5e10}{cycles}$
  \item{P3:} $\SI{4.0}{\giga\hertz}\times\SI{10}{\second} = \SI{4e10}{cycles}$
  \end{itemize}

  $$\text{instructions} = \frac{\text{cycles}}{\text{CPI}}$$

  \begin{itemize}
  \item{P1:} $\SI{3e10}{} / \SI{1.5}{} = \SI{2e10}{instructions}$
  \item{P2:} $\SI{2.5e10}{} / \SI{1.0}{} = \SI{2.5e10}{instructions}$
  \item{P3:} $\SI{4e10}{} / \SI{2.2}{} = \SI{1.818e10}{instructions}$
  \end{itemize}

\item
  \begin{fancyquotes}
    We are trying to reduce the time by 30\% but this leads to an
    increase of 20\% in the CPI\@. What clock rate should we have to get
    this time reduction?
  \end{fancyquotes}

  $$\text{instructions} = \frac{\text{cycles}}{\text{CPI}}
  = \frac{\text{clock rate}\times\text{time}}{\text{CPI}}$$
  $$\text{clock rate}
  = \frac{\text{instructions}\times\text{CPI}}{\text{time}}$$
  $$\text{clock rate}^{\prime}
  = \frac{\text{instructions}\times\text{CPI}^{\prime}}{\text{time}^{\prime}}$$
  $$\text{CPI}^{\prime} = 1.2\text{CPI}$$
  $$\text{time}^{\prime} = 0.7\text{time}$$

  So:
  $$\frac{\text{clock rate}^{\prime}}{\text{clock rate}}
  = \frac{\text{CPI}^{\prime}}{\text{CPI}}\times
  \frac{\text{time}^{\prime}}{\text{time}}
  = 1.2 / 0.7 = \frac{12}{7}$$

  For processors:
  \begin{itemize}
  \item{P1:} $\SI{3.0}{\giga\hertz}\times\frac{12}{7} =
    \SI{5.1429}{\giga\hertz}$
  \item{P2:} $\SI{2.5}{\giga\hertz}\times\frac{12}{7} =
    \SI{4.2857}{\giga\hertz}$
  \item{P3:} $\SI{4.0}{\giga\hertz}\times\frac{12}{7} =
    \SI{6.8571}{\giga\hertz}$
  \end{itemize}

\end{enumerate}


\pagebreak

% -----------------------------------------------------------------------------
% PROBLEM 2
% -----------------------------------------------------------------------------
\section{}

\begin{fancyquotes}
  1.6 [20] <§1.6>

  Consider two different implementations of the same instruction set
  architecture. The instructions can be divided into four classes
  according to their CPI (class A, B, C, and D). $P1$ with a clock rate
  of $\SI{2.5}{\giga\hertz}$ and CPIs of 1, 2, 3, and 3, and $P2$ with
  a clock rate of $\SI{3}{\giga\hertz}$ and CPIs of 2, 2, 2, and 2.

  Given a program with a dynamic instruction count of
  $\SI{1.0E6}{instructions}$ divided into classes as follows: $10\%$
  class A, $20\%$ class B, $50\%$ class C, and $20\%$ class D, which
  implementation is faster?
\end{fancyquotes}

\begin{enumerate}
\item
  \begin{fancyquotes}
    What is the global CPI for each implementation?
  \end{fancyquotes}

  \begin{itemize}
  \item{P1:} $\text{CPI} = 10\%\times 1 + 20\%\times 2 + 50\%\times 3
    + 20\%\times 3 = 2.6$
  \item{P2:} $\text{CPI} = 10\%\times 2 + 20\%\times 2 + 50\%\times 2
    + 20\%\times 2 = 2$
  \end{itemize}

\item
  \begin{fancyquotes}
    Find the clock cycles required in both cases.”
  \end{fancyquotes}

  $$\text{cycles} = \text{CPI}\times\text{instructions}$$

  \begin{itemize}
  \item{P1:} $2.6\times\SI{1.0e6}{} = \SI{2.6e6}{cycles}$
  \item{P2:} $2.0\times\SI{1.0e6}{} = \SI{2.0e6}{cycles}$
  \end{itemize}
\end{enumerate}

So for running time:

$$\text{time} = \frac{\text{cycles}}{\text{clock rate}}$$

\begin{itemize}
\item{P1:} $\SI{2.6e6}{} / \SI{2.5}{} = \SI{1.04}{\second}$
\item{P2:} $\SI{2.0e6}{} / \SI{3.0}{} = \SI{0.6667}{\second}$
\end{itemize}

$P2$ is faster.

\pagebreak

% -----------------------------------------------------------------------------
% PROBLEM 3
% -----------------------------------------------------------------------------
\section{}

\begin{fancyquotes}
  1.7 [15] <§1.6>

  Compilers can have a profound impact on the performance of an
  application. Assume that for a program, compiler $A$ results in a
  dynamic instruction count of $\SI{1.0e9}{}$ and has an execution
  time of $\SI{1.1}{\second}$, while compiler $B$ results in a dynamic
  instruction count of $\SI{1.2e9}{}$ and an execution time of
  $\SI{1.5}{\second}$.
\end{fancyquotes}

\begin{enumerate}
\item
  \begin{fancyquotes}
    Find the average CPI for each program given that the processor has
    a clock cycle time of $\SI{1}{\nano\second}$.
  \end{fancyquotes}

  $$\text{CPI} = \frac{\text{cycles}}{\text{instructions}}
  = \frac{\text{running time} /
    \text{cycle time}}{\text{instructions}}$$

  \begin{itemize}
  \item{A:} $\frac{\SI{1.1}{\second} /
      \SI{1}{\nano\second}}{\SI{1.0e9}{}} = 1.1$
  \item{B:} $\frac{\SI{1.5}{\second} /
      \SI{1}{\nano\second}}{\SI{1.2e9}{}} = 1.25$
  \end{itemize}

\item
  \begin{fancyquotes}
    Assume the compiled programs run on two different processors. If
    the execution times on the two processors are the same, how much
    faster is the clock of the processor running compiler $A$'s code
    versus the clock of the processor running compiler $B$'s code?
  \end{fancyquotes}

  $$\text{clock rate} =
  \frac{\text{CPI}\times\text{instructions}}{\text{running time}}$$

  \begin{itemize}
  \item{A:} $1.1\times\SI{1.0e9} = \SI{1.1}{\giga\hertz}$
  \item{B:} $1.25\times\SI{1.2e9} = \SI{1.5}{\giga\hertz}$
  \end{itemize}

\item
  \begin{fancyquotes}
    A new compiler is developed that uses only $\SI{6.0e8}{}$
    instructions and has an average CPI of $1.1$. What is the speedup
    of using this new compiler versus using compiler $A$ or $B$ on the
    original processor?
  \end{fancyquotes}

  $$\text{running time}
  = \frac{\text{CPI}\times\text{instructions}}{\text{clock rate}}$$
  $$\frac{\text{running speed}^\prime}{\text{running speed}}
  = \frac{\text{running time}}{\text{running time}^{\prime}}
  = \frac{\text{CPI}\times\text{instructions}}
  {\text{CPI}^\prime\times\text{instructions}^\prime}$$

  \begin{itemize}
  \item{C vs A:}
    $\frac{1.1\times \SI{1.0e9}{}} {1.1\times \SI{6e8}{}} = 1.6667$
    About $66.67\%$ faster.
  \item{C vs B:}
    $\frac{1.25\times \SI{1.2e9}{}} {1.1\times \SI{6e8}{}} = 2.2727$
    About $127.27\%$ faster.
  \end{itemize}

\end{enumerate}

\end{document}
