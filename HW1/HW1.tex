\documentclass[paper=a4, fontsize=11pt]{scrartcl} % A4 paper and 11pt font size
\usepackage{./../usfassignment}
\usepackage{siunitx}
\settitle{Assignment 1}
\setauthor{Wanzhang Sheng}
\setcourse{CS315: Computer Architecture}

\begin{document}

\maketitle % Print the title

% -----------------------------------------------------------------------------
% PROBLEM 1
% -----------------------------------------------------------------------------
\section{1.5}

\begin{fancyquotes}
  Consider three different processors $P1$, $P2$, and $P3$ executing
  the same instruction set. $P1$ has a $\SI{3}{\giga\hertz}$ clock
  rate and a CPI of $1.5$. $P2$ has a $\SI{2.5}{\giga\hertz}$ clock
  rate and a CPI of $1.0$. $P3$ has a $\SI{4.0}{\giga\hertz}$ clock
  rate and has a CPI of $2.2$.
\end{fancyquotes}

\subsection{1.5.a}
\begin{fancyquotes}
  Which processor has the highest performance expressed in
  instructions per second?
\end{fancyquotes}

$$\text{Instructions} / \text{second} = \frac{\text{clock rate}}{\text{CPI}}$$

\begin{itemize}
\item{P1:} $\SI{3.0}{\giga\hertz} / 1.5 = \SI{2e9}{}$
\item{P2:} $\SI{2.5}{\giga\hertz} / 1.0 = \SI{2.5e9}{}$
\item{P3:} $\SI{4.0}{\giga\hertz} / 2.2 = \SI{1.818e9}{}$
\end{itemize}

So $P2$ has the hightest performance in instructions per second.

\subsection{1.5.b}
\begin{fancyquotes}
  If the processors each execute a program in $\SI{10}{\second}$,
  find the number of cycles and the number of instructions.
\end{fancyquotes}

$$\text{cycles} = \text{clock rate} \times \text{time in seconds}$$

\begin{itemize}
\item{P1:} $\SI{3.0}{\giga\hertz}\times\SI{10}{\second} = \SI{3e10}{cycles}$
\item{P2:} $\SI{2.5}{\giga\hertz}\times\SI{10}{\second} = \SI{2.5e10}{cycles}$
\item{P3:} $\SI{4.0}{\giga\hertz}\times\SI{10}{\second} = \SI{4e10}{cycles}$
\end{itemize}

$$\text{instructions} = \frac{\text{cycles}}{\text{CPI}}$$

\begin{itemize}
\item{P1:} $\SI{3e10}{} / \SI{1.5}{} = \SI{2e10}{instructions}$
\item{P2:} $\SI{2.5e10}{} / \SI{1.0}{} = \SI{2.5e10}{instructions}$
\item{P3:} $\SI{4e10}{} / \SI{2.2}{} = \SI{1.818e10}{instructions}$
\end{itemize}

\subsection{1.5.c}
\begin{fancyquotes}
  We are trying to reduce the time by 30\% but this leads to an
  increase of 20\% in the CPI\@. What clock rate should we have to get
  this time reduction?
\end{fancyquotes}

$$\text{instructions} = \frac{\text{cycles}}{\text{CPI}}
= \frac{\text{clock rate}\times\text{time}}{\text{CPI}}$$
$$\text{clock rate}
= \frac{\text{instructions}\times\text{CPI}}{\text{time}}$$
$$\text{clock rate}^{\prime}
= \frac{\text{instructions}\times\text{CPI}^{\prime}}{\text{time}^{\prime}}$$
$$\text{CPI}^{\prime} = 1.2\text{CPI}$$
$$\text{time}^{\prime} = 0.7\text{time}$$

So:
$$\frac{\text{clock rate}^{\prime}}{\text{clock rate}}
= \frac{\text{CPI}^{\prime}}{\text{CPI}}\times
\frac{\text{time}^{\prime}}{\text{time}}
= 1.2 / 0.7 = \frac{12}{7}$$

For processors:
\begin{itemize}
\item{P1:} $\SI{3.0}{\giga\hertz}\times\frac{12}{7} =
  \SI{5.1429}{\giga\hertz}$
\item{P2:} $\SI{2.5}{\giga\hertz}\times\frac{12}{7} =
  \SI{4.2857}{\giga\hertz}$
\item{P3:} $\SI{4.0}{\giga\hertz}\times\frac{12}{7} =
  \SI{6.8571}{\giga\hertz}$
\end{itemize}

\pagebreak

% -----------------------------------------------------------------------------
% PROBLEM 2
% -----------------------------------------------------------------------------
\section{1.6}

\begin{fancyquotes}
  Consider two different implementations of the same instruction set
  architecture. The instructions can be divided into four classes
  according to their CPI (class A, B, C, and D). $P1$ with a clock rate
  of $\SI{2.5}{\giga\hertz}$ and CPIs of 1, 2, 3, and 3, and $P2$ with
  a clock rate of $\SI{3}{\giga\hertz}$ and CPIs of 2, 2, 2, and 2.

  Given a program with a dynamic instruction count of
  $\SI{1.0E6}{instructions}$ divided into classes as follows: $10\%$
  class A, $20\%$ class B, $50\%$ class C, and $20\%$ class D, which
  implementation is faster?
\end{fancyquotes}

\subsection{1.6.a}
\begin{fancyquotes}
  What is the global CPI for each implementation?
\end{fancyquotes}

\begin{itemize}
\item{P1:} $\text{CPI} = 10\%\times 1 + 20\%\times 2 + 50\%\times 3
  + 20\%\times 3 = 2.6$
\item{P2:} $\text{CPI} = 10\%\times 2 + 20\%\times 2 + 50\%\times 2
  + 20\%\times 2 = 2$
\end{itemize}

\subsection{1.6.b}
\begin{fancyquotes}
  Find the clock cycles required in both cases.”
\end{fancyquotes}

$$\text{cycles} = \text{CPI}\times\text{instructions}$$

\begin{itemize}
\item{P1:} $2.6\times\SI{1.0e6}{} = \SI{2.6e6}{cycles}$
\item{P2:} $2.0\times\SI{1.0e6}{} = \SI{2.0e6}{cycles}$
\end{itemize}

So for running time:

$$\text{time} = \frac{\text{cycles}}{\text{clock rate}}$$

\begin{itemize}
\item{P1:} $\SI{2.6e6}{} / \SI{2.5}{} = \SI{1.04}{\second}$
\item{P2:} $\SI{2.0e6}{} / \SI{3.0}{} = \SI{0.6667}{\second}$
\end{itemize}

$P2$ is faster.

\pagebreak

% -----------------------------------------------------------------------------
% PROBLEM 3
% -----------------------------------------------------------------------------
\section{1.7}

\begin{fancyquotes}
  Compilers can have a profound impact on the performance of an
  application. Assume that for a program, compiler $A$ results in a
  dynamic instruction count of $\SI{1.0e9}{}$ and has an execution
  time of $\SI{1.1}{\second}$, while compiler $B$ results in a dynamic
  instruction count of $\SI{1.2e9}{}$ and an execution time of
  $\SI{1.5}{\second}$.
\end{fancyquotes}

\subsection{1.7.a}
\begin{fancyquotes}
  Find the average CPI for each program given that the processor has
  a clock cycle time of $\SI{1}{\nano\second}$.
\end{fancyquotes}

$$\text{CPI} = \frac{\text{cycles}}{\text{instructions}}
= \frac{\text{running time} /
  \text{cycle time}}{\text{instructions}}$$

\begin{itemize}
\item{A:} $\frac{\SI{1.1}{\second} /
    \SI{1}{\nano\second}}{\SI{1.0e9}{}} = 1.1$
\item{B:} $\frac{\SI{1.5}{\second} /
    \SI{1}{\nano\second}}{\SI{1.2e9}{}} = 1.25$
\end{itemize}

\subsection{1.7.b}
\begin{fancyquotes}
  Assume the compiled programs run on two different processors. If
  the execution times on the two processors are the same, how much
  faster is the clock of the processor running compiler $A$'s code
  versus the clock of the processor running compiler $B$'s code?
\end{fancyquotes}

$$\text{clock rate} =
\frac{\text{CPI}\times\text{instructions}}{\text{running time}}$$

\begin{itemize}
\item{A:} $1.1\times\SI{1.0e9} = \SI{1.1}{\giga\hertz}$
\item{B:} $1.25\times\SI{1.2e9} = \SI{1.5}{\giga\hertz}$
\end{itemize}

\subsection{1.7.c}
\begin{fancyquotes}
  A new compiler is developed that uses only $\SI{6.0e8}{}$
  instructions and has an average CPI of $1.1$. What is the speedup
  of using this new compiler versus using compiler $A$ or $B$ on the
  original processor?
\end{fancyquotes}

$$\frac{\text{running speed}^\prime}{\text{running speed}}
= \frac{\text{running time}}{\text{running time}^{\prime}}
= \frac{\frac{\text{CPI}\times\text{instructions}}{\text{clock rate}}}
{\frac{\text{CPI}^\prime\times\text{instructions}^\prime}{\text{clock rate}}}
= \frac{\text{CPI}\times\text{instructions}}
{\text{CPI}^\prime\times\text{instructions}^\prime}$$

\begin{itemize}
\item{C vs A:}
  $\frac{1.1\times \SI{1.0e9}{}} {1.1\times \SI{6e8}{}} = 1.6667$
  About $66.67\%$ faster.
\item{C vs B:}
  $\frac{1.25\times \SI{1.2e9}{}} {1.1\times \SI{6e8}{}} = 2.2727$
  About $127.27\%$ faster.
\end{itemize}

\pagebreak

% -----------------------------------------------------------------------------
% PROBLEM 4
% -----------------------------------------------------------------------------
\section{1.9}

\begin{fancyquotes}
  Assume for arithmetic, load/store, and branch instructions, a
  processor has CPIs of $1$, $12$, and $5$, respectively. Also assume
  that on a single processor a program requires the execution of
  $\SI{2.56E9}{}$ arithmetic instructions, $\SI{1.28E9}{}$ load/store
  instructions, and $256$ million branch instructions. Assume that
  each processor has a $\SI{2}{\giga\hertz}$ clock frequency.

  Assume that, as the program is parallelized to run over multiple
  cores, the number of arithmetic and load/store instructions per
  processor is divided by $0.7\times p$ (where $p$ is the number of
  processors) but the number of branch instructions per processor
  remains the same.
\end{fancyquotes}

\subsection{1.9.1}
\begin{fancyquotes}
  Find the total execution time for this program on $1$, $2$, $4$,
  and $8$ processors, and show the relative speedup of the $2$, $4$,
  and $8$ processor result relative to the single processor result.
\end{fancyquotes}

$$\text{running time} =
\frac{\text{instructions}\times\text{CPI}}{\text{clock rate}}$$

\begin{itemize}
\item{1:} $(\SI{2.56e9}{}\times 1 + \SI{1.28e9}{}\times 12 +
  \SI{256e6}{}\times 5) \div \SI{2}{\giga\hertz} = \SI{9.6}{\second}$
\item{2:} $(\frac{\SI{2.56e9}{}}{0.7\times 2}\times 1 +
  \frac{\SI{1.28e9}{}}{0.7\times 2}\times 12 + \SI{256e6}{}\times 5)
  \div \SI{2}{\giga\hertz} = \SI{7.019}{\second}$\\
  ratio vs single: $73.11\%$
\item{4:} $(\frac{\SI{2.56e9}{}}{0.7\times 4}\times 1 +
  \frac{\SI{1.28e9}{}}{0.7\times 4}\times 12 + \SI{256e6}{}\times 5)
  \div \SI{2}{\giga\hertz} = \SI{3.829}{\second}$\\
  ratio vs single: $39.89\%$
\item{8:} $(\frac{\SI{2.56e9}{}}{0.7\times 8}\times 1 +
  \frac{\SI{1.28e9}{}}{0.7\times 8}\times 12 + \SI{256e6}{}\times 5)
  \div \SI{2}{\giga\hertz} = \SI{2.235}{\second}$\\
  ratio vs single: $23.28\%$
\end{itemize}

\subsection{1.9.2}
\begin{fancyquotes}
  If the CPI of the arithmetic instructions was doubled, what would
  the impact be on the execution time of the program on $1$, $2$,
  $4$, or $8$ processors?
\end{fancyquotes}

\begin{itemize}
\item{1:} $(\SI{2.56e9}{}\times 2 + \SI{1.28e9}{}\times 12 +
  \SI{256e6}{}\times 5) \div \SI{2}{\giga\hertz} = \SI{10.88}{\second}$
\item{2:} $(\frac{\SI{2.56e9}{}}{0.7\times 2}\times 2 +
  \frac{\SI{1.28e9}{}}{0.7\times 2}\times 12 + \SI{256e6}{}\times 5)
  \div \SI{2}{\giga\hertz} = \SI{7.954}{\second}$\\
  ratio vs single: $73.11\%$
\item{4:} $(\frac{\SI{2.56e9}{}}{0.7\times 4}\times 2 +
  \frac{\SI{1.28e9}{}}{0.7\times 4}\times 12 + \SI{256e6}{}\times 5)
  \div \SI{2}{\giga\hertz} = \SI{4.297}{\second}$\\
  ratio vs single: $39.50\%$
\item{8:} $(\frac{\SI{2.56e9}{}}{0.7\times 8}\times 2 +
  \frac{\SI{1.28e9}{}}{0.7\times 8}\times 12 + \SI{256e6}{}\times 5)
  \div \SI{2}{\giga\hertz} = \SI{2.469}{\second}$\\
  ratio vs single: $22.69\%$
\end{itemize}

\subsection{1.9.3}
\begin{fancyquotes}
  To what should the CPI of load/store instructions be reduced in
  order for a single processor to match the performance of four
  processors using the original CPI values?
\end{fancyquotes}

Assume the expected CPI of load/store instructions is $x$.

$$(\SI{2.56e9}{}\times 2 + \SI{1.28e9}{}\times x +
\SI{256e6}{}\times 5)
\div \SI{2}{\giga\hertz}
= \SI{3.829}{\second}$$
$$x = 0.9828125$$

\pagebreak

% -----------------------------------------------------------------------------
% PROBLEM 5
% -----------------------------------------------------------------------------
\section{1.13}

\begin{fancyquotes}
  Another pitfall cited in Section 1.10 is expecting to improve the
  overall performance of a computer by improving only one aspect of
  the computer. Consider a computer running a program that requires
  $\SI{250}{\second}$, with $\SI{70}{\second}$ spent executing FP
  instructions, $\SI{85}{\second}$ executed L/S instructions, and
  $\SI{40}{\second}$ spent executing branch instructions.
\end{fancyquotes}

\subsection{1.13.1}
\begin{fancyquotes}
  By how much is the total time reduced if the time for FP operations
  is reduced by 20\%?
\end{fancyquotes}

Total time reduced $70\times 20\% = \SI{14}{\second}$.

\subsection{1.13.3}
\begin{fancyquotes}
  Can the total time can be reduced by 20\% by reducing only the time
  \% for branch instructions?
\end{fancyquotes}

No. Assume the branch instructions time is $\SI{0}{\second}$.
The total time reduced is $1 - \frac{250-40}{250} = 16\% < 20\%$.
So it's impossible.

\pagebreak

% -----------------------------------------------------------------------------
% PROBLEM  6
% -----------------------------------------------------------------------------
\section{1.15}

\begin{fancyquotes}
  When a program is adapted to run on multiple processors in a
  multiprocessor system, the execution time on each processor is
  comprised of computing time and the overhead time required for
  locked critical sections and/or to send data from one processor to
  another.

  Assume a program requires $t=\SI{100}{\second}$ of execution time on
  one processor. When run $p$ processors, each processor requires
  $t/p$, as well as an additional $\SI{4}{\second}$ of
  overhead, irrespective of the number of processors. Compute the
  per-processor execution time for 2, 4, 8, 16, 32, 64, and 128
  processors. For each case, list the corresponding speedup relative
  to a single processor and the ratio between actual speedup versus
  ideal speedup (speedup if there was no overhead).
\end{fancyquotes}

\begin{itemize}
\item{1:} $\SI{100}{\second}$
\item{2:} $100\div 2 + 4 = \SI{54}{\second}$\\
  speedup: $100\div 54 - 1 = 85.19\%$ vs $100\%$
\item{4:} $100\div 4 + 4 = \SI{29}{\second}$\\
  speedup: $100\div 29 - 1 = 244.83\%$ vs $300\%$
\item{8:} $100\div 8 + 4 = \SI{16.5}{\second}$\\
  speedup: $100\div 16.5 - 1 = 506.06\%$ vs $700\%$
\item{16:} $100\div 16 + 4 = \SI{10.25}{\second}$\\
  speedup: $100\div 10.25 - 1 = 875.61\%$ vs $1600\%$
\item{32:} $100\div 32 + 4 = \SI{7.125}{\second}$\\
  speedup: $100\div 7.125 - 1 = 1303.51\%$ vs $3200\%$
\item{64:} $100\div 64 + 4 = \SI{5.563}{\second}$\\
  speedup: $100\div 5.563 - 1 = 1697.59\%$ vs $6400\%$
\item{128:} $100\div 128 + 4 = \SI{4.781}{\second}$\\
  speedup: $100\div 4.781 - 1 = 1991.61\%$ vs $12800\%$
\end{itemize}

\pagebreak

\end{document}
